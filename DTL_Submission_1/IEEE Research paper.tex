\documentclass[conference]{IEEEtran}
\IEEEoverridecommandlockouts
% The preceding line is only needed to identify funding in the first footnote. If that is unneeded, please comment it out.
\usepackage{cite}
\usepackage{amsmath,amssymb,amsfonts}
\usepackage{algorithmic}
\usepackage{graphicx}
\usepackage{textcomp}
\usepackage{xcolor}
\def\BibTeX{{\rm B\kern-.05em{\sc i\kern-.025em b}\kern-.08em
    T\kern-.1667em\lower.7ex\hbox{E}\kern-.125emX}}

\begin{document}

\title{IEEE Research Paper\\


}

\author{\IEEEauthorblockN{1\textsuperscript{st} Sahil Deshpande}
\IEEEauthorblockA{\textit{Computer Engineering Department)} \\
\textit{name of organization (COEP)}\\
Pune, India \\
email address or ORCID}
\and
\IEEEauthorblockN{2\textsuperscript{nd} Sameer Deshpande}
\IEEEauthorblockA{\textit{Computer Engineering Department} \\
\textit{name of organization (COEP)}\\
Pune, India \\
email address or ORCID}
\and
}
\maketitle

\begin{abstract}
This document is a model and instructions for \LaTeX.
This and the IEEEtran.cls file define the components of your paper [title, text, heads, etc.]. *CRITICAL: Do Not Use Symbols, Special Characters, Footnotes, 
or Math in Paper Title or Abstract.
\end{abstract}

\begin{IEEEkeywords}
component, formatting, style, styling, insert
\end{IEEEkeywords}

\section{Introduction}
This document is a model and instructions for \LaTeX.
Please observe the conference page limits. 

\section{Ease of Use}

\subsection{Maintaining the Integrity of the Specifications}

The IEEEtran class file is used to format your paper and style the text. All margins, 
column widths, line spaces, and text fonts are prescribed; please do not 
alter them. You may note peculiarities. For example, the head margin
measures proportionately more than is customary. This measurement 
and others are deliberate, using specifications that anticipate your paper 
as one part of the entire proceedings, and not as an independent document. 
Please do not revise any of the current designations.

\section{Prepare Your Paper Before Styling}
Before you begin to format your paper, first write and save the content as a 
separate text file. Complete all content and organizational editing before 
formatting. Please note sections \ref{AA}--\ref{SCM} below for more information on 
proofreading, spelling and grammar.

Keep your text and graphic files separate until after the text has been 
formatted and styled. Do not number text heads---{\LaTeX} will do that 
for you.

\subsection{Abbreviations and Acronyms}\label{AA}
Define abbreviations and acronyms the first time they are used in the text, 
even after they have been defined in the abstract. Abbreviations such as 
IEEE, SI, MKS, CGS, ac, dc, and rms do not have to be defined. Do not use 
abbreviations in the title or heads unless they are unavoidable.

\subsection{Units}
\begin{itemize}
\item Use either SI (MKS) or CGS as primary units. (SI units are encouraged.) English units may be used as secondary units (in parentheses). An exception would be the use of English units as identifiers in trade, such as ``3.5-inch disk drive''.
\item Avoid combining SI and CGS units, such as current in amperes and magnetic field in oersteds. This often leads to confusion because equations do not balance dimensionally. If you must use mixed units, clearly state the units for each quantity that you use in an equation.
\item Do not mix complete spellings and abbreviations of units: ``Wb/m\textsuperscript{2}'' or ``webers per square meter'', not ``webers/m\textsuperscript{2}''. Spell out units when they appear in text: ``. . . a few henries'', not ``. . . a few H''.
\item Use a zero before decimal points: ``0.25'', not ``.25''. Use ``cm\textsuperscript{3}'', not ``cc''.)
\end{itemize}

\subsection{Gravitation}


Newton's law of universal gravitation is usually stated as that every particle attracts every other particle in the universe with a force that is proportional to the product of their masses and inversely proportional to the square of the distance between their centers.


\subsection{Concorde}
The Aérospatiale/BAC Concorde is a retired Franco-British supersonic airliner jointly developed and manufactured by Sud Aviation and the British Aircraft Corporation (BAC). Studies started in 1954, and France and the UK signed a treaty establishing the development project on 29 November 1962, as the programme cost was estimated at £70 million (£1.39 billion in 2021). Construction of the six prototypes began in February 1965, and the first flight took off from Toulouse on 2 March 1969. The market was predicted for 350 aircraft, and the manufacturers received up to 100 option orders from many major airlines. On 9 October 1975, it received its French Certificate of Airworthiness, and from the UK CAA on 5 December.

\subsection{Mars Mission}\label{SCM}

The Mars Orbiter Mission (MOM), also called Mangalyaan, was a space probe orbiting Mars since 24 September 2014. It was launched on 5 November 2013 by the Indian Space Research Organisation (ISRO). It was India's first interplanetary mission and it made ISRO the fourth space agency to achieve Mars orbit, after Roscosmos, NASA, and the European Space Agency. It made India the first Asian nation to reach the Martian orbit and the first nation in the world to do so on its maiden attempt
An excellent style manual for science writers is \cite{b7}.

\subsection{Authors and Affiliations}
\textbf{The class file is designed for, but not limited to, six authors.} A 
minimum of one author is required for all conference articles. Author names 
should be listed starting from left to right and then moving down to the 
next line. This is the author sequence that will be used in future citations 
and by indexing services. Names should not be listed in columns nor group by 
affiliation. Please keep your affiliations as succinct as possible (for 
example, do not differentiate among departments of the same organization).

\subsection{JEE Mains}
The Joint Entrance Examination (JEE) is an engineering entrance assessment conducted for admission to various engineering colleges in India. It is constituted by two different examinations: the JEE-Main and the JEE-Advanced.

The Joint Seat Allocation Authority (JoSAA) conducts the joint admission process for a total of 23 Indian Institutes of Technology, 31 National Institutes of Technology, 25 Indian Institutes of Information Technology campuses and other Government Funded Technical Institutes (GFTIs) based on the rank obtained by a student in JEE-Main or JEE-Advanced, depending on the engineering college.

\subsection{Figures and Tables}
\paragraph{Positioning Figures and Tables} Place figures and tables at the top and 
bottom of columns. Avoid placing them in the middle of columns. Large 
figures and tables may span across both columns. Figure captions should be 
below the figures; table heads should appear above the tables. Insert 
figures and tables after they are cited in the text. Use the abbreviation 
``Fig.~\ref{fig}'', even at the beginning of a sentence.

\begin{table}[h!]
\caption{Table Type Styles}
\begin{center}
\begin{tabular}{|l|c|c|c|}
\hline
\textbf{Name} & \textbf{Coaching} & \textbf{Percentile} & \textbf{College}\\
\hline
Snehasish & ICAD & 99.84 & COEP\\
\hline
Sahil & Cataylser & 99.77 & COEP\\
\hline
John & FITJEE & 100 & IIT Bombay\\
\hline
James & Allen & 99.22 & VNIT\\
\hline




\end{tabular}
\end{center}
\end{table}

\begin{figure}[htbp]

\caption{Example of a figure caption.}
\label{fig}
\end{figure}


\pagebreak
\section*{Acknowledgment}

The preferred spelling of the word ``acknowledgment'' in America is without 
an ``e'' after the ``g''. Avoid the stilted expression ``one of us (R. B. 
G.) thanks $\ldots$''. Instead, try ``R. B. G. thanks$\ldots$''. Put sponsor 
acknowledgments in the unnumbered footnote on the first page.

\section*{References}

Please number citations consecutively within brackets \cite{b1}. The 
sentence punctuation follows the bracket \cite{b2}. Refer simply to the reference 
number, as in \cite{b3}---do not use ``Ref. \cite{b3}'' or ``reference \cite{b3}'' except at 
the beginning of a sentence: ``Reference \cite{b3} was the first $\ldots$''

Number footnotes separately in superscripts. Place the actual footnote at 
the bottom of the column in which it was cited. Do not put footnotes in the 
abstract or reference list. Use letters for table footnotes.

Unless there are six authors or more give all authors' names; do not use 
``et al.''. Papers that have not been published, even if they have been 
submitted for publication, should be cited as ``unpublished'' \cite{b4}. Papers 
that have been accepted for publication should be cited as ``in press'' \cite{b5}. 
Capitalize only the first word in a paper title, except for proper nouns and 
element symbols.

For papers published in translation journals, please give the English 
citation first, followed by the original foreign-language citation \cite{b6}.

\begin{thebibliography}{00}
\bibitem{b1} G. Eason, B. Noble, and I. N. Sneddon, ``On certain integrals of Lipschitz-Hankel type involving products of Bessel functions,'' Phil. Trans. Roy. Soc. London, vol. A247, pp. 529--551, April 1955.
\bibitem{b2} J. Clerk Maxwell, A Treatise on Electricity and Magnetism, 3rd ed., vol. 2. Oxford: Clarendon, 1892, pp.68--73.
\bibitem{b3} I. S. Jacobs and C. P. Bean, ``Fine particles, thin films and exchange anisotropy,'' in Magnetism, vol. III, G. T. Rado and H. Suhl, Eds. New York: Academic, 1963, pp. 271--350.
\bibitem{b4} K. Elissa, ``Title of paper if known,'' unpublished.
\bibitem{b5} R. Nicole, ``Title of paper with only first word capitalized,'' J. Name Stand. Abbrev., in press.
\bibitem{b6} Y. Yorozu, M. Hirano, K. Oka, and Y. Tagawa, ``Electron spectroscopy studies on magneto-optical media and plastic substrate interface,'' IEEE Transl. J. Magn. Japan, vol. 2, pp. 740--741, August 1987 [Digests 9th Annual Conf. Magnetics Japan, p. 301, 1982].
\bibitem{b7} M. Young, The Technical Writer's Handbook. Mill Valley, CA: University Science, 1989.
\end{thebibliography}


\end{document}